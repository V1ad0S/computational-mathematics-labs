\documentclass[fleqn]{article}
\usepackage[top = 15 mm, left = 15 mm, right = 15 mm, bottom = 20 mm]{geometry}

\usepackage[T2A]{fontenc}			% кодировка
\usepackage[utf8]{inputenc}			% кодировка исходного текста
\usepackage[english,russian]{babel}	% локализация и переносы

\usepackage{amsmath,amsfonts,amssymb,amsthm,mathtools}

\usepackage[usenames]{color}
\usepackage{colortbl}
\usepackage{indentfirst} %Красная строка
\usepackage{hyperref}

\begin{document}

Многочлен Ньютона имеет следующий вид
\[ P_N(x) = \sum_{n = 0}^N b_n \prod_{i = 0}^{n-1} (x - x_i) \]

Рассмотрим задачу о приведении многочлена Ньютона к степенному виду:
\[ P_N(x) = \sum_{i = 0}^N A_{N_i} x^i \]

Выражения $P_N$ при $N \in \{0, 1, 2, 3\}$
\begin{align*}
	P_0(x) &= b_0 \\
	P_1(x) &= b_0 + b_1 (x - x_0) = b_1 x + (b_0 - b_1 x_0) \\
	P_2(x) &= b_0 + b_1 (x - x_0) + b_2 (x - x_0)(x - x_1) =
	b_2 x^2 + (b_1 - b_2 (x_0 + x_1)) x + (b_0 - b_1 x_0 + b_2 x_0 x_1) \\
	P_3(x) &= b_3 x^3 + (b_2 - b_3 (x_0 + x_1 + x_2)) x^2
	+ (b_1 - b_2 (x_0 + x_1) + b_3 (x_0 x_1 + x_0 x_2 + x_1 x_2)) x + \\
	&+ (b_0 - b_1 x_0 + b_2 x_0 x_1 - b_3 x_0 x_1 x_2)
\end{align*}


Выпишем свободные коэффициенты при разных $N$:
\begin{align*}
	P_0 &: A_{0_0} = b_0 \\
	P_1 &: A_{1_0} = b_0 - b_1 x_0 \\
	P_2 &: A_{2_0} = b_0 - b_1 x_0 + b_2 x_0 x_1 \\
	P_3 &: A_{3_0} = b_0 - b_1 x_0 + b_2 x_0 x_1 - b_3 x_0 x_1 x_2
\end{align*}

Легко заметить, что с ростом $N$ каждый следующий коэффициент частично
повторяет предыдущий:
\[ A_{N_0} = A_{N-1_0} + b_N \cdot f_0(x_0, \ldots , x_{N-1}) \]

Оказывается такое соотношение выполняется не только для свободного члена,
но и для всех коэффициентов степенного полинома. Составим матрицу:
\[
	M = 
\begin{pmatrix*}[l]
	b_0 & 0 & 0 & 0\\
	M_{00} + b_1 \cdot f_0 (x_0) & b_1 & 0 & 0\\
	M_{10} + b_2 \cdot f_0 (x_0, x_1) & M_{11} + b_2 \cdot f_1(x_0, x_1) & b_2 & 0\\
	M_{20} + b_3 \cdot f_0 (x_0, x_1, x_2) & M_{21} + b_3 \cdot f_1(x_0, x_1, x_2) &
		M_{22} + b_3 \cdot f_2(x_0, x_1, x_2) & b_3
\end{pmatrix*}
\]

Матрица составлена так, что элемент $M_{ij}$ --- это коэффициент перед $x^j$
в полиноме Ньютона степени $i$.

Теперь необходимо определить функции $f_i$; $f_0$ определяется довольно просто с точностью
до знака, из ранее выписанных свободных коэффициентов. Это просто произведение всех аргументов:
\[f_0(x_0, \ldots , x_n) = \prod_{i = 0}^n x_i = x_0 \cdot x_1 \cdot \ldots \cdot x_n\]


Для $f_1$ имеем следующие равенства:
\begin{align*}
	&b_2 \cdot f_1(x_0, x_1) = b_2 (x_0 + x_1)\\
	&b_3 \cdot f_1(x_0, x_1, x_2) = b_3 (x_0 x_1 + x_0 x_2 + x_1 x_2)
\end{align*}

Получаем выражение для $f_1(x_0, x_1, x_2)$ через $f_1(x_0, x_1)$ и $f_0(x_0, x_1)$:
\[ f_1(x_0, x_1, x_2) = (x_0 x_1 + x_0 x_2 + x_1 x_2) = x_2 (x_0 + x_1) + x_0 x_1 = x_2 \cdot f_1(x_0, x_1) + f_0(x_0, x_1) \]

Аналогично получаются выражения для $\{f_i\}_{i=2}^N$.
Можно записать значения функций $\{f_i\}$ в виде матрицы:
\[
	F = 
\begin{pmatrix*}[l]
	x_0					& 1									& 0									& 0\\
	F_{00} \cdot x_1	& F_{01} \cdot x_1 + F_{00}	& 1									& 0\\
	F_{10} \cdot x_2	& F_{11} \cdot x_2 + F_{10}	& F_{12} \cdot x_2 + F_{11}	& 1\\
	F_{20} \cdot x_3	& F_{21} \cdot x_3 + F_{20}	& F_{22} \cdot x_3 + F_{21}	& F_{23} \cdot x_3 + F_{22}
\end{pmatrix*}
\]

Теперь значение функции $f_i(x_0, x_1, \ldots , x_j) = F_{ji}$.


\end{document}